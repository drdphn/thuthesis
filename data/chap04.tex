\chapter{研究方案}
\section{研究设计}

\subsection{总体研究设计}
本研究采用实证研究与案例分析相结合的方法,旨在通过网络分析优化新产品导入(NPI)流程中的任务依赖、资源分配和风险管理。首先,将通过文献综述和理论分析明确研究框架,进而通过实际案例的深入分析,验证网络分析方法在NPI流程中的有效性。最后,将提出具体的优化策略,并通过模拟或实地验证来检验这些策略的实际应用效果。

\subsection{研究对象选择}
本研究的研究对象将选取在高科技制造业中的一个或多个新产品导入项目。选择这些项目的原因在于高科技制造业中产品更新换代速度快,NPI过程复杂,涉及多部门、多任务的协同工作。这种复杂的项目环境为网络分析方法的应用提供了良好的研究基础。此外,高科技制造业企业通常拥有较为成熟的项目管理体系和丰富的数据资源,有助于网络分析模型的构建与验证。

\section{研究方法}

\subsection{文献综述与理论框架构建}
首先,通过对国内外相关文献的系统梳理,明确网络分析在NPI流程管理中的应用现状与挑战,并结合复杂性管理理论,构建本研究的理论框架。这一过程包括以下步骤:
\begin{enumerate}
    \item 系统检索和收集网络分析、复杂性管理、NPI流程管理等领域的相关文献。
    \item 对收集到的文献进行分类和分析,识别关键理论和方法。
    \item 基于文献分析结果,构建本研究的理论框架,明确研究假设和问题。
\end{enumerate}

\subsection{案例研究与数据收集}
案例研究是本研究的核心部分,通过对一个或多个实际NPI项目的详细分析,收集所需的数据,以支持网络分析模型的构建。具体方法包括:
\begin{itemize}
    \item \textbf{数据收集}:通过企业内部访谈、项目文档分析和实地调研等方式,收集与NPI项目相关的任务依赖关系、资源分配情况和项目进度数据。这些数据将用于构建任务依赖网络。
    \item \textbf{案例分析}:对收集到的数据进行初步分析,识别项目中的关键任务、瓶颈问题和资源分配不均等现象,为后续的网络建模和分析奠定基础。
\end{itemize}

\subsection{网络建模与分析}
基于所收集的数据,利用网络分析工具(如Gephi、Pajek等)构建NPI流程的任务依赖网络。具体步骤如下:
\begin{enumerate}
    \item \textbf{网络构建}:将NPI项目中的任务视为节点,任务之间的依赖关系视为边,构建任务依赖网络图。
    \item \textbf{关键路径分析}:利用网络分析方法(如度中心性、介数中心性等)识别网络中的关键路径和瓶颈任务。
    \item \textbf{资源分配分析}:通过分析资源节点在网络中的位置和作用,评估现有资源分配的合理性,并提出优化建议。
    \item \textbf{风险识别与管理}:识别网络中存在的潜在风险点,并基于网络结构提出相应的风险管理策略。
\end{enumerate}

\subsection{模拟验证与优化策略}
在网络分析完成后,本研究将通过模拟或实际应用来验证提出的优化策略。具体步骤包括:
\begin{itemize}
    \item \textbf{模拟实验}:使用模拟工具(如AnyLogic、Simul8等)模拟不同优化方案在NPI流程中的效果,比较其对项目进度、资源利用率和风险管理的影响。
    \item \textbf{实际验证}:在合作企业中选择一个实际的NPI项目应用优化策略,观察其在真实环境中的表现,收集反馈数据进行进一步分析。
    \item \textbf{优化策略调整}:根据模拟和实际验证的结果,进一步调整和完善优化策略,确保其在不同项目环境中的适用性和有效性。
\end{itemize}

\section{技术路线}
为了更好地描述研究过程,本研究的技术路线可以通过以下图示表示:
% \begin{figure}[h]
%     \centering
%     \includegraphics[width=0.8\textwidth]{tech_route.png}
%     \caption{研究技术路线图}
%     \label{fig:tech_route}
% \end{figure}

该技术路线图展示了从文献综述、理论分析,到案例研究、数据收集,再到网络建模、分析及优化验证的整个研究过程。各个步骤之间相互关联,确保研究的系统性和严谨性。

\section{本章小结}
本章详细介绍了本研究的总体设计、研究方法及技术路线。通过文献综述、案例研究、网络分析和模拟验证的综合应用,本研究旨在优化NPI流程中的任务依赖关系和资源分配,提出切实可行的优化策略,为企业的新产品导入过程提供理论指导和实践参考。
