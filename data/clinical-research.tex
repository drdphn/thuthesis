\bibliographyunit[\chapter]

\renewcommand{\chaptermark}[1]{\markboth{#1}{}}
\chapter*{\mdseries 因素A对因素B的影响}

\addcontentsline{toc}{chapter}{临床研究}
\markboth{临床研究}{}

\newcommand{\chineseHead}[1]{{\heiti #1}}
\newcommand{\englishTitle}[1]{{\sffamily\centering\Large #1\par}}

% \vspace{24pt}
% \chineseTitle{粮食安全对早期肺功能异常的影响}
% \vspace{24pt} % Space before abstract

% Chinese abstract
\noindent\chineseHead{【摘要】}
黑体,小四号字,左右实心凸形括号。摘要内容书写,宋体, 小四号,两端对齐,字符间距为 常规(标准) 。行距为固定值20磅,段前空0磅, 段后空0磅。
\\ \hspace*{\fill} \\
\noindent\chineseHead{【关键词】}
因素A;因素B;关系C

\vspace{18pt} % Space after Chinese part

\englishTitle{Impact of factor A to factor B}
\vspace{24pt} % Space before abstract

\noindent\textbf{Abstract:} Factor A is xxx
\\ \hspace*{\fill} \\
\noindent\textbf{Keywords:} factorA, factorB, relationC
English abstract


\renewcommand{\thesection}{\arabic{section}}
\setcounter{section}{0}
\newpage
\section{Introduction}

Factor A is a xxx

\section{Method}

We used xxx

\section{Results}

We found...

\begin{figure}
    \centering
    \includegraphics[width=1\linewidth]{example-image-a.pdf}
    \caption{Title and Number of figure in Arial}
    \caption*{This legend is in Times Roman}
  \end{figure}

\section{Discussion}

In this research xxx

In summary xxx

\section*{参考文献}
{\renewcommand{\bibsection}{}
\putbib
}
