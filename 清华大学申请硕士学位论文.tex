% !TeX encoding = UTF-8
% !TeX program = xelatex
% !TeX spellcheck = en_US

\documentclass[degree=master, degree-type=professional]{thuthesis}
  % 学位 degree:
  %   doctor | master | bachelor | postdoc
  % 学位类型 degree-type:
  %   academic(默认)| professional
  % 语言 language
  %   chinese(默认)| english
  % 字体库 fontset
  %   windows | mac | fandol | ubuntu
  % 建议终版使用 Windows 平台的字体编译


% 论文基本配置,加载宏包等全局配置
% !TeX root = ./thuthesis-example.tex

% 论文基本信息配置

\thusetup{
  %******************************
  % 注意:
  %   1. 配置里面不要出现空行
  %   2. 不需要的配置信息可以删除
  %   3. 建议先阅读文档中所有关于选项的说明
  %******************************
  %
  % 输出格式
  %   选择打印版(print)或用于提交的电子版(electronic),前者会插入空白页以便直接双面打印
  %
  output = print,
  %
  % 标题
  %   可使用“\\”命令手动控制换行
  %
  title  = {清华大学学位论文 \LaTeX{} 模板\\使用示例文档 v\version},
  title* = {An Introduction to \LaTeX{} Thesis Template of Tsinghua
            University v\version},
  %
  % 学位
  %   1. 学术型
  %      - 中文
  %        需注明所属的学科门类,例如:
  %        哲学、经济学、法学、教育学、文学、历史学、理学、工学、农学、医学、
  %        军事学、管理学、艺术学
  %      - 英文
  %        博士:Doctor of Philosophy
  %        硕士:
  %          哲学、文学、历史学、法学、教育学、艺术学门类,公共管理学科
  %          填写“Master of Arts“,其它填写“Master of Science”
  %   2. 专业型
  %      直接填写专业学位的名称,例如:
  %      教育博士、工程硕士等
  %      Doctor of Education, Master of Engineering
  %   3. 本科生不需要填写
  %
  degree-name  = {工学硕士},
  degree-name* = {Master of Science},
  %
  % 培养单位
  %   填写所属院系的全名
  %
  department = {计算机科学与技术系},
  %
  % 学科
  %   1. 学术型学位
  %      获得一级学科授权的学科填写一级学科名称,其他填写二级学科名称
  %   2. 专业学位
  %      (a) 设置专业领域的专业学位,填写专业领域名称
  %      (b) 2019 级及之前工程硕士学位,在 `engineering-field` 填写工程领域名称
  %      (c) 其他专业学位,无需此信息
  %   3. 本科生填写专业名称,第二学位论文需标注“(第二学位)”
  %
  discipline  = {计算机科学与技术},
  discipline* = {Computer Science and Technology},
  %
  % 姓名
  %
  author  = {薛瑞尼},
  author* = {Xue Ruini},
  %
  % 指导教师
  %   中文姓名和职称之间以英文逗号“,”分开,下同
  %
  supervisor  = {郑纬民, 教授},
  supervisor* = {Professor Zheng Weimin},
  %
  % 副指导教师
  %
  associate-supervisor  = {陈文光, 教授},
  associate-supervisor* = {Professor Chen Wenguang},
  %
  % 联合指导教师
  %
  % co-supervisor  = {某某某, 教授},
  % co-supervisor* = {Professor Mou Moumou},
  %
  % 日期
  %   使用 ISO 格式;默认为当前时间
  %
  % date = {2019-07-07},
  %
  % 是否在中文封面后的空白页生成书脊(默认 false)
  %
  include-spine = false,
  %
  % 密级和年限
  %   秘密, 机密, 绝密
  %
  % secret-level = {秘密},
  % secret-year  = {10},
  %
  % 博士后专有部分
  %
  % clc                = {分类号},
  % udc                = {UDC},
  % id                 = {编号},
  % discipline-level-1 = {计算机科学与技术},  % 流动站(一级学科)名称
  % discipline-level-2 = {系统结构},          % 专业(二级学科)名称
  % start-date         = {2011-07-01},        % 研究工作起始时间
}

% 载入所需的宏包

% 定理类环境宏包
\usepackage{amsthm}
% 也可以使用 ntheorem
% \usepackage[amsmath,thmmarks,hyperref]{ntheorem}

\thusetup{
  %
  % 数学字体
  % math-style = GB,  % GB | ISO | TeX
  math-font  = xits,  % stix | xits | libertinus
}

% 可以使用 nomencl 生成符号和缩略语说明
% \usepackage{nomencl}
% \makenomenclature

% 表格加脚注
\usepackage{threeparttable}

% 表格中支持跨行
\usepackage{multirow}

% 固定宽度的表格。
% \usepackage{tabularx}

% 跨页表格
\usepackage{longtable}

% 算法
\usepackage{algorithm}
\usepackage{algorithmic}

% 量和单位
\usepackage{siunitx}

% 参考文献使用 BibTeX + natbib 宏包
% 顺序编码制
\usepackage[sort]{natbib}
\bibliographystyle{thuthesis-numeric}

% 著者-出版年制
% \usepackage{natbib}
% \bibliographystyle{thuthesis-author-year}

% 本科生参考文献的著录格式
% \usepackage[sort]{natbib}
% \bibliographystyle{thuthesis-bachelor}

% 参考文献使用 BibLaTeX 宏包
% \usepackage[style=thuthesis-numeric]{biblatex}
% \usepackage[style=thuthesis-author-year]{biblatex}
% \usepackage[style=apa]{biblatex}
% \usepackage[style=mla-new]{biblatex}
% \usepackage[notes,backend=biber]{biblatex-chicago}
% 声明 BibLaTeX 的数据库
% \addbibresource{ref/refs.bib}

% 参考文献使用 citation-style-language 宏包(试验性)
% 需要 TeX Live 2023+ 以及最新版 citation-style-language 宏包
% \usepackage{citation-style-language}
% \cslsetup{style = thuthesis-numeric}
% \cslsetup{style = thuthesis-author-year}
% \cslsetup{style = apa}
% \cslsetup{style = modern-language-association}
% \cslsetup{style = cell}  % 生物医学
% \cslsetup{style = tsinghua-university-academy-of-arts-and-design}  % 美术学院
% 声明 CSL 的数据库
% \addbibresource{ref/refs.bib}

% 定义所有的图片文件在 figures 子目录下
\graphicspath{{figures/}}

% 数学命令
\makeatletter
\newcommand\dif{%  % 微分符号
  \mathop{}\!%
  \ifthu@math@style@TeX
    d%
  \else
    \mathrm{d}%
  \fi
}
\makeatother

% hyperref 宏包在最后调用
\usepackage{hyperref}



\begin{document}

% 封面
\maketitle

% 学位论文指导小组、公开评阅人和答辩委员会名单
% 本科生不需要
% % !TeX root = ../thuthesis-example.tex

\begin{committee}[name={毕业论文公开评阅人和答辩委员会名单}]

  \newcolumntype{C}[1]{@{}>{\centering\arraybackslash}p{#1}}

  \section*{公开评阅人名单}

  \begin{center}
    \begin{tabular}{C{3cm}C{3cm}C{9cm}@{}}
      刘XX & 教授   & 清华大学                    \\
      陈XX & 副教授 & XXXX大学                    \\
      杨XX & 研究员 & 中国XXXX科学院XXXXXXX研究所 \\
    \end{tabular}
  \end{center}


  \section*{答辩委员会名单}

  \begin{center}
    \begin{tabular}{C{2.75cm}C{2.98cm}C{4.63cm}C{4.63cm}@{}}
      主席 & 赵XX                  & 教授                    & 清华大学       \\
      委员 & 刘XX                  & 教授                    & 清华大学       \\
          & \multirow{2}{*}{杨XX} & \multirow{2}{*}{研究员} & 中国XXXX科学院 \\
          &                       &                         & XXXXXXX研究所  \\
          & 黄XX                  & 教授                    & XXXX大学       \\
          & 周XX                  & 副教授                  & XXXX大学       \\
      秘书 & 吴XX                  & 助理研究员              & 清华大学       \\
    \end{tabular}
  \end{center}

\end{committee}



% 也可以导入 Word 版转的 PDF 文件
% \begin{committee}[file=figures/committee.pdf]
% \end{committee}


% 使用授权的说明
% 本科生开题报告不需要
% \copyrightpage
% 将签字扫描后授权文件 scan-copyright.pdf 替换原始页面
% \copyrightpage[file=scan-copyright.pdf]

% \frontmatter
% \input{data/abstract}

% 目录
\tableofcontents

% 插图和附表清单
% 本科生的插图索引和表格索引需要移至正文之后、参考文献前
% \listoffiguresandtables  % 插图和附表清单(仅限研究生)
% \listoffigures           % 插图清单
% \listoftables            % 附表清单

% 符号对照表
% \input{data/denotation}


% 正文部分
\mainmatter
% !TeX root = ../thuthesis-example.tex

\chapter{选题背景}

\section{Apple Watch 简介}

Apple Watch 是苹果公司于2015年推出的一款智能手表,迅速成为全球可穿戴设备市场的领导者。
Apple Watch 集合了健康监测、通讯、支付等多种功能,旨在提供便捷、高效和个性化的用户体验。
其在健康和健身追踪、心率监测、ECG 功能和血氧检测等方面的创新,
使其成为健康管理领域的重要工具\citep{apple2024, gehani2016corporate, davidson_assessing_2023}。

Apple Watch 的设计和功能不仅局限于技术创新,还包括用户体验和市场需求的综合考虑。
苹果公司通过与医疗技术公司合作,引入了先进的健康监测技术,
并与多方合作伙伴共同开发了专门针对运动爱好者的Apple Watch Nike+系列\citep{apple2024}。
此外,Apple Watch 的开放平台策略鼓励第三方开发者为其开发应用,
进一步丰富了其功能和生态系统\citep{dahlander_how_2010}。

Apple Watch 在市场上的成功也与其多功能集成和用户参与密切相关。
通过收集用户反馈和参与,苹果公司能够及时了解市场需求,并根据反馈进行产品改进和功能更新,
从而保持产品的市场竞争力\citep{chesbrough_beyond_2006, chesbrough2016}。
Apple Watch 的多功能性和高用户满意度使其成为市场上备受推崇的智能手表,
推动了可穿戴设备市场的快速发展\citep{gehani2016corporate, davidson_assessing_2023}。

\section{开放式创新的概念}

开放式创新是一种新兴的创新模式,强调企业在创新过程中不仅依赖内部资源,
还广泛利用外部资源。开放式创新的核心理念是通过合作、技术引进和用户参与等方式,
打破企业内部与外部的边界,提升创新能力\citep{chesbrough_beyond_2006}。
通过与外部合作伙伴、客户、大学和研究机构等合作,企业可以更快速地获取新技术、开拓新市场,
并有效降低研发成本和风险\citep{dahlander_how_2010}。

开放式创新模式的优势在于它能够有效整合内外部资源,加速创新过程。
例如,苹果公司通过与全球供应链伙伴的紧密合作,不仅在技术创新方面取得了显著成果,
还提升了产品的市场竞争力\citep{zhang_pingguo_2014}. 此外,
开放式创新还能够通过引入外部创意和技术,弥补企业内部资源和能力的不足,
增强企业的整体创新能力\citep{chen_ziyuan_2006}. 

在开放式创新模式下,企业不仅仅是技术的消费者,也是技术的提供者。
企业可以通过技术授权、合作研发等方式,将内部的技术和创新成果输出到市场,
从而实现技术价值的最大化\citep{lu_chanpinzhiliang_2021}. 例如,一些制造企业通过开放式创新,
不仅提升了产品质量,还增强了市场竞争力\citep{han_xinchanpin_2021}.

此外,开放式创新在不同行业中的应用也各具特色。在高科技行业,开放式创新能够通过跨领域的技术合作,
促进技术的快速转移和应用\citep{dong_gongyinglian_2024}. 而在传统制造业,
开放式创新则更多地体现在通过引入外部技术和管理经验,提高生产效率和产品质量\citep{sun_neiwiajianxiu_2024}.

开放式创新不仅仅是技术层面的创新,更是一种管理模式的创新。
它要求企业在创新管理中具备开放的心态和灵活的机制,能够有效地识别、
吸收和整合外部的创新资源\citep{zhang_shizheng_2012}.
 这对于企业的创新文化和组织能力提出了更高的要求。


\section{Apple Watch 开发中的开放式创新}

在Apple Watch的开发过程中,苹果公司采用了多种开放式创新策略。
例如,苹果公司通过与外部医疗技术公司合作,引入先进的健康监测技术;
通过开放平台,鼓励第三方开发者为Apple Watch开发应用,丰富其生态系统;
通过收购具有关键技术的公司,快速获取技术资源,增强产品竞争力\citep{apple2024,davidson_assessing_2023}。

\section{研究选题的意义和重要性}

研究Apple Watch在开发过程中的开放式创新应用,不仅有助于深入了解苹果公司的创新机制,
还能为其他科技企业提供有价值的借鉴。通过系统分析Apple Watch的开放式创新策略,
可以揭示其成功背后的关键因素,帮助其他企业在产品开发中更好地利用开放式创新。
同时,随着科技产品的快速迭代和市场竞争的加剧,理解开放式创新在产品开发中的应用
也具有重要的理论和实践意义\citep{gehani2016corporate,chesbrough2022}。

\section{本研究的目标}

本研究旨在探讨Apple Watch开发过程中的开放式创新应用,具体目标包括:
\begin{itemize}
    \item 分析Apple Watch的开发过程,梳理其关键里程碑和创新点。
    \item 探讨苹果公司在Apple Watch开发过程中采用的开放式创新策略。
    \item 评估开放式创新对Apple Watch技术创新、市场表现和用户满意度的影响。
    \item 提出开放式创新在可穿戴设备开发中的最佳实践和建议。
\end{itemize}
\chapter{国内外研究现状}

\section{国外研究现状}

\subsection{网络分析理论的发展}
网络分析(Network Analysis)作为一种研究复杂网络结构及其动态行为的方法,最早起源于社会网络分析,随后逐渐扩展到物理学、计算机科学和生物学等多个领域\cite{Newman2003Structure}。在20世纪末和21世纪初,随着图论和计算技术的发展,Barabási等人提出了“无尺度网络”模型,揭示了许多复杂网络的幂律分布特性\cite{Barabasi2002Linked}。这一理论的发展为网络分析在各类复杂系统中的应用奠定了基础。

在项目管理领域,网络分析逐渐被应用于流程优化和风险管理等方面。国外学者通过将网络分析方法引入到新产品导入(NPI)过程中,旨在解决传统项目管理方法在面对多任务依赖和复杂性管理时的不足。Kerzner等人\cite{Kerzner2017Project}指出,通过网络分析可以有效识别项目中的关键路径和瓶颈任务,从而优化项目进度并提高资源利用效率。此外,国外研究还探讨了如何结合大数据和人工智能技术,进一步增强网络分析在项目管理中的应用能力。

\subsection{新产品导入过程中的复杂性管理}
NPI过程中的复杂性管理一直是国外学术界和企业界关注的重点。由于NPI过程涉及多个部门的协同工作和大量的资源调配,传统的线性项目管理方法难以应对其复杂性和动态性。国外研究通过引入复杂性科学的理论,提出了多种管理方法,如基于网络分析的复杂性管理框架,以及结合系统动力学的混合模型\cite{Sterman2000SystemDynamics},这些研究为复杂性管理提供了新的思路。

在实际应用中,国外企业已经开始尝试将这些理论应用于实践。例如,IBM、通用电气等跨国公司在其产品导入过程中,广泛应用了网络分析技术,用于任务调度、风险评估和资源优化配置。这些企业通过构建和分析NPI流程中的任务依赖网络,识别出了关键任务和潜在风险,并相应调整了项目管理策略,显著提高了项目的成功率。

\subsection{新产品开发中的网络分析}
Pflaum 和 Weissenberger-Eibl 的研究\cite{pflaum2017using}探讨了网络分析在新产品开发(NPD)中的应用,旨在通过识别系统行为模式和关键变量来提高工程性能。研究表明,网络分析可以帮助研发经理识别对新产品成功至关重要的因素,从而提高产品设计和开发的效率。

\subsection{复杂网络在产品开发中的结构分析}
Batallas 和 Yassine 的研究\cite{batallas2006information}通过社会网络分析探讨了大规模产品开发网络中的无标度结构(Scale-Free structure)。他们确定了在复杂产品开发组织网络中起关键作用的信息领导者,并提供了管理这些复杂网络的建议,从而优化了产品开发过程中的信息流管理。

\subsection{基于复杂网络理论的产品开发过程建模}
Bencherif 和 Mouss 提出了一种基于复杂网络理论的产品开发过程模型\cite{bencherif2020complex},强调了在建模产品开发过程中对创新环境和战略框架的表征分析。这项研究展示了复杂网络在增强产品开发过程建模和策略优化方面的重要性。

\subsection{结论}
综上所述,网络分析为产品设计与开发中的多个关键领域提供了强有力的支持工具。从提升决策效率到优化信息流管理,网络分析在理解和管理复杂系统方面展示了其不可或缺的价值。


\section{国内研究现状}

\subsection{国内网络分析应用研究}
国内对网络分析的研究起步较晚,但近年来在复杂性管理、供应链管理等领域取得了显著进展。国内学者逐渐认识到网络分析在处理复杂系统中的优势,并将其应用于不同的管理场景。在项目管理领域,国内研究开始关注如何将网络分析引入到NPI过程中,以应对项目管理中的复杂性和多变性问题。

与国外相比,国内的研究更多地聚焦于网络分析的理论探讨和方法改进。近年来,一些研究者开始尝试将网络分析与机器学习、数据挖掘等新兴技术结合,以提高其在复杂项目管理中的应用效果。然而,国内在网络分析的实际应用方面,尤其是在NPI过程中的应用研究仍相对较少,缺乏系统的实证研究和案例分析。

\subsection{NPI过程管理的国内现状}
在新产品导入的研究领域,国内学者主要集中在传统项目管理方法的优化上,如关键路径法(CPM)和项目评估与审查技术(PERT)等。这些方法在处理小规模项目和单一任务依赖时表现良好,但在面对复杂的NPI流程时,表现出了一定的局限性。随着市场竞争的加剧和产品开发周期的缩短,如何有效管理NPI过程中的复杂性和不确定性,成为了国内研究亟待解决的问题。

近年来,国内一些大型企业,尤其是高科技和制造业企业,开始引入国外的先进管理方法和技术,如网络分析和大数据分析,用于优化其产品导入流程。 然而,这些应用仍处于探索阶段,缺乏系统的理论指导和全面的实践验证。因此,国内对NPI复杂性管理的研究亟需进一步深入,特别是在结合实际应用和理论创新方面。

\section{小结}
通过对国内外研究现状的分析,可以发现国外在网络分析和NPI过程复杂性管理方面的研究已经取得了较为成熟的成果,且在实际应用中取得了一定的成效。相比之下,国内在这一领域的研究仍处于起步和探索阶段,尤其是在实证研究和应用推广方面与国外存在一定差距。因此,本研究将结合国外的先进经验,基于国内企业的实际需求,深入探讨网络分析在NPI过程中的应用,力求为国内NPI过程管理提供理论支持和实践指导。


% !TeX root = ../thuthesis-example.tex

\chapter{数学符号和公式}

\section{数学符号}

中文论文的数学符号默认遵循 GB/T 3102.11—1993《物理科学和技术中使用的数学符号》
\footnote{原 GB 3102.11—1993,自 2017 年 3 月 23 日起,该标准转为推荐性标准。}。
该标准参照采纳 ISO 31-11:1992 \footnote{目前已更新为 ISO 80000-2:2019。},
但是与 \TeX{} 默认的美国数学学会(AMS)的符号习惯有所区别。
具体地来说主要有以下差异:
\begin{enumerate}
  \item 大写希腊字母默认为斜体,如
    \begin{equation*}
      \Gamma \Delta \Theta \Lambda \Xi \Pi \Sigma \Upsilon \Phi \Psi \Omega.
    \end{equation*}
    注意有限增量符号 $\increment$ 固定使用正体,模板提供了 \cs{increment} 命令。
  \item 小于等于号和大于等于号使用倾斜的字形 $\le$、$\ge$。
  \item 积分号使用正体,比如 $\int$、$\oint$。
  \item
    偏微分符号 $\partial$ 使用正体。
  \item
    省略号 \cs{dots} 按照中文的习惯固定居中,比如
    \begin{equation*}
      1, 2, \dots, n \quad 1 + 2 + \dots + n.
    \end{equation*}
  \item
    实部 $\Re$ 和虚部 $\Im$ 的字体使用罗马体。
\end{enumerate}

以上数学符号样式的差异可以在模板中统一设置。
另外国标还有一些与 AMS 不同的符号使用习惯,需要用户在写作时进行处理:
\begin{enumerate}
  \item 数学常数和特殊函数名用正体,如
    \begin{equation*}
      \uppi = 3.14\dots; \quad
      \symup{i}^2 = -1; \quad
      \symup{e} = \lim_{n \to \infty} \left( 1 + \frac{1}{n} \right)^n.
    \end{equation*}
  \item 微分号使用正体,比如 $\dif y / \dif x$。
  \item 向量、矩阵和张量用粗斜体(\cs{symbf}),如 $\symbf{x}$、$\symbf{\Sigma}$、$\symbfsf{T}$。
  \item 自然对数用 $\ln x$ 不用 $\log x$。
\end{enumerate}


英文论文的数学符号使用 \TeX{} 默认的样式。
如果有必要,也可以通过设置 \verb|math-style| 选择数学符号样式。

关于量和单位推荐使用
\href{http://mirrors.ctan.org/macros/latex/contrib/siunitx/siunitx.pdf}{\pkg{siunitx}}
宏包,
可以方便地处理希腊字母以及数字与单位之间的空白,
比如:
\SI{6.4e6}{m},
\SI{9}{\micro\meter},
\si{kg.m.s^{-1}},
\SIrange{10}{20}{\degreeCelsius}。



\section{数学公式}

数学公式可以使用 \env{equation} 和 \env{equation*} 环境。
注意数学公式的引用应前后带括号,通常使用 \cs{eqref} 命令,比如式\eqref{eq:example}。
\begin{equation}
  \frac{1}{2 \uppi \symup{i}} \int_\gamma f = \sum_{k=1}^m n(\gamma; a_k) \mathscr{R}(f; a_k).
  \label{eq:example}
\end{equation}

多行公式尽可能在“=”处对齐,推荐使用 \env{align} 环境。
\begin{align}
  a & = b + c + d + e \\
    & = f + g
\end{align}



\section{数学定理}

定理环境的格式可以使用 \pkg{amsthm} 或者 \pkg{ntheorem} 宏包配置。
用户在导言区载入这两者之一后,模板会自动配置 \env{theorem}、\env{proof} 等环境。

\begin{theorem}[Lindeberg--Lévy 中心极限定理]
  设随机变量 $X_1, X_2, \dots, X_n$ 独立同分布, 且具有期望 $\mu$ 和有限的方差 $\sigma^2 \ne 0$,
  记 $\bar{X}_n = \frac{1}{n} \sum_{i+1}^n X_i$,则
  \begin{equation}
    \lim_{n \to \infty} P \left(\frac{\sqrt{n} \left( \bar{X}_n - \mu \right)}{\sigma} \le z \right) = \Phi(z),
  \end{equation}
  其中 $\Phi(z)$ 是标准正态分布的分布函数。
\end{theorem}
\begin{proof}
  Trivial.
\end{proof}

同时模板还提供了 \env{assumption}、\env{definition}、\env{proposition}、
\env{lemma}、\env{theorem}、\env{axiom}、\env{corollary}、\env{exercise}、
\env{example}、\env{remar}、\env{problem}、\env{conjecture} 这些相关的环境。

\chapter{研究方案}
\section{研究设计}

\subsection{总体研究设计}
本研究采用实证研究与案例分析相结合的方法,旨在通过网络分析优化新产品导入(NPI)流程中的任务依赖、资源分配和风险管理。首先,将通过文献综述和理论分析明确研究框架,进而通过实际案例的深入分析,验证网络分析方法在NPI流程中的有效性。最后,将提出具体的优化策略,并通过模拟或实地验证来检验这些策略的实际应用效果。

\subsection{研究对象选择}
本研究的研究对象将选取在高科技制造业中的一个或多个新产品导入项目。选择这些项目的原因在于高科技制造业中产品更新换代速度快,NPI过程复杂,涉及多部门、多任务的协同工作。这种复杂的项目环境为网络分析方法的应用提供了良好的研究基础。此外,高科技制造业企业通常拥有较为成熟的项目管理体系和丰富的数据资源,有助于网络分析模型的构建与验证。

\section{研究方法}

\subsection{文献综述与理论框架构建}
首先,通过对国内外相关文献的系统梳理,明确网络分析在NPI流程管理中的应用现状与挑战,并结合复杂性管理理论,构建本研究的理论框架。这一过程包括以下步骤:
\begin{enumerate}
    \item 系统检索和收集网络分析、复杂性管理、NPI流程管理等领域的相关文献。
    \item 对收集到的文献进行分类和分析,识别关键理论和方法。
    \item 基于文献分析结果,构建本研究的理论框架,明确研究假设和问题。
\end{enumerate}

\subsection{案例研究与数据收集}
案例研究是本研究的核心部分,通过对一个或多个实际NPI项目的详细分析,收集所需的数据,以支持网络分析模型的构建。具体方法包括:
\begin{itemize}
    \item \textbf{数据收集}:通过企业内部访谈、项目文档分析和实地调研等方式,收集与NPI项目相关的任务依赖关系、资源分配情况和项目进度数据。这些数据将用于构建任务依赖网络。
    \item \textbf{案例分析}:对收集到的数据进行初步分析,识别项目中的关键任务、瓶颈问题和资源分配不均等现象,为后续的网络建模和分析奠定基础。
\end{itemize}

\subsection{网络建模与分析}
基于所收集的数据,利用网络分析工具(如Gephi、Pajek等)构建NPI流程的任务依赖网络。具体步骤如下:
\begin{enumerate}
    \item \textbf{网络构建}:将NPI项目中的任务视为节点,任务之间的依赖关系视为边,构建任务依赖网络图。
    \item \textbf{关键路径分析}:利用网络分析方法(如度中心性、介数中心性等)识别网络中的关键路径和瓶颈任务。
    \item \textbf{资源分配分析}:通过分析资源节点在网络中的位置和作用,评估现有资源分配的合理性,并提出优化建议。
    \item \textbf{风险识别与管理}:识别网络中存在的潜在风险点,并基于网络结构提出相应的风险管理策略。
\end{enumerate}

\subsection{模拟验证与优化策略}
在网络分析完成后,本研究将通过模拟或实际应用来验证提出的优化策略。具体步骤包括:
\begin{itemize}
    \item \textbf{模拟实验}:使用模拟工具(如AnyLogic、Simul8等)模拟不同优化方案在NPI流程中的效果,比较其对项目进度、资源利用率和风险管理的影响。
    \item \textbf{实际验证}:在合作企业中选择一个实际的NPI项目应用优化策略,观察其在真实环境中的表现,收集反馈数据进行进一步分析。
    \item \textbf{优化策略调整}:根据模拟和实际验证的结果,进一步调整和完善优化策略,确保其在不同项目环境中的适用性和有效性。
\end{itemize}

\section{技术路线}
为了更好地描述研究过程,本研究的技术路线可以通过以下图示表示:
% \begin{figure}[h]
%     \centering
%     \includegraphics[width=0.8\textwidth]{tech_route.png}
%     \caption{研究技术路线图}
%     \label{fig:tech_route}
% \end{figure}

该技术路线图展示了从文献综述、理论分析,到案例研究、数据收集,再到网络建模、分析及优化验证的整个研究过程。各个步骤之间相互关联,确保研究的系统性和严谨性。

\section{本章小结}
本章详细介绍了本研究的总体设计、研究方法及技术路线。通过文献综述、案例研究、网络分析和模拟验证的综合应用,本研究旨在优化NPI流程中的任务依赖关系和资源分配,提出切实可行的优化策略,为企业的新产品导入过程提供理论指导和实践参考。

\chapter{预期成果及可能的创新点}

\section{预期成果}

\subsection{理论成果}
本研究预计将在以下几个方面对现有理论作出贡献:
\begin{itemize}
    \item \textbf{网络分析在NPI流程管理中的应用拓展}:通过将网络分析方法引入NPI流程管理,丰富了网络分析在项目管理领域的应用场景,特别是在复杂性管理和资源优化方面的研究。
    \item \textbf{复杂性管理的理论深化}:本研究将复杂性管理的理论与实际项目管理过程结合,进一步深化了复杂性管理的应用研究,为应对复杂系统中的动态管理挑战提供了新的视角。
    \item \textbf{项目管理工具的集成应用}:本研究探索了如何将网络分析与传统项目管理工具(如关键路径法、PERT等)进行集成,为项目管理工具的协同使用提供了理论基础和实证支持。
\end{itemize}

\subsection{实践成果}
在实践层面,本研究将为企业提供以下可操作的成果:
\begin{itemize}
    \item \textbf{NPI流程优化建议报告}:基于网络分析结果,提出针对企业实际需求的NPI流程优化策略,帮助企业提升项目管理效率和资源利用率,缩短产品上市时间。
    \item \textbf{任务依赖和资源分配优化模型}:构建一个适用于NPI流程的任务依赖和资源分配优化模型,为企业提供一种科学的方法来识别和管理项目中的关键任务和资源瓶颈。
    \item \textbf{风险管理策略}:提出通过网络分析识别和管理NPI过程中潜在风险的策略,帮助企业有效应对项目中的不确定性,降低项目失败的风险。
\end{itemize}

\section{可能的创新点}

\subsection{理论创新}
\begin{itemize}
    \item \textbf{网络分析与复杂性管理的结合}:本研究首次将网络分析与复杂性管理理论系统结合,提出了一种新型的项目管理方法论。这种方法不仅可以识别NPI流程中的关键路径和瓶颈任务,还可以动态优化资源分配和风险管理,具有较高的理论创新价值。
    \item \textbf{项目管理方法的扩展}:传统的项目管理方法在处理复杂任务依赖时往往存在局限性,本研究通过引入网络分析,拓展了项目管理方法的适用范围,尤其是在多任务、多团队的复杂项目环境中。
\end{itemize}

\subsection{实践创新}
\begin{itemize}
    \item \textbf{应用场景的拓展}:本研究将网络分析方法应用于NPI流程优化,开创了该方法在高科技制造业项目管理中的新应用场景。这为未来更多企业在新产品导入过程中采用网络分析提供了参考案例。
    \item \textbf{优化模型的创新应用}:通过构建具体的任务依赖和资源分配优化模型,本研究为企业提供了实际可操作的工具,帮助企业在复杂的NPI过程中实现高效管理。这一模型具有广泛的应用潜力,可推广至其他复杂项目管理领域。
    \item \textbf{实践验证的创新}:本研究不仅提出了理论框架和方法,还通过实际案例验证了这些方法的有效性,为网络分析在项目管理中的应用提供了坚实的实践基础。这种理论与实践相结合的研究路径在NPI管理领域具有开创性。
\end{itemize}

\section{本章小结}
本章总结了本研究的预期成果及可能的创新点。理论上,本研究将丰富和拓展网络分析在NPI流程管理中的应用,为复杂性管理提供新的视角和方法论支持;实践上,本研究为企业提供了实用的NPI流程优化工具和风险管理策略。这些成果不仅在学术上具有重要意义,也为实际项目管理提供了切实可行的指导。


% !TeX root = ../thuthesis-example.tex

\chapter{研究计划}

为确保本研究的顺利进行和高质量完成,制定了详细的研究计划和时间表。研究计划涵盖了文献收集、数据分析、报告撰写和修改等多个阶段。具体计划如下:

\section{研究阶段与时间安排}

\begin{itemize}
    \item \textbf{前期准备}(2024年7月 - 2024年9月)
    \begin{itemize}
        \item 确定研究主题和研究问题。
        \item 收集和整理相关文献,进行文献综述。
        \item 制定详细的研究计划和方法。
    \end{itemize}

    \item \textbf{数据收集}(2024年10月 - 2024年12月)
    \begin{itemize}
        \item 收集与Apple Watch开发过程相关的案例和数据。
        \item 查找和整理苹果公司发布的合作公告、专利信息和用户反馈。
    \end{itemize}

    \item \textbf{数据分析}(2025年1月 - 2025年3月)
    \begin{itemize}
        \item 进行定性分析,总结开放式创新策略的实施效果和经验教训。
        \item 利用统计方法,对用户反馈和市场数据进行定量分析,评估开放式创新对产品成功的具体影响。
    \end{itemize}

    \item \textbf{撰写报告}(2025年4月 - 2025年6月)
    \begin{itemize}
        \item 整理和分析研究数据,撰写研究结果和讨论部分。
        \item 完成研究报告的初稿,并进行内部审阅和修改。
    \end{itemize}

    \item \textbf{评审与修改}(2025年7月)
    \begin{itemize}
        \item 根据导师和评审专家的反馈,对研究报告进行修改和完善。
        \item 准备最终版本的研究报告,进行论文答辩准备。
    \end{itemize}
\end{itemize}

\section{关键任务和目标}

为了确保各阶段任务的顺利完成,每个阶段都设定了具体的目标和关键任务:

\subsection{前期准备阶段}

\begin{itemize}
    \item \textbf{目标}:明确研究主题和问题,奠定研究基础。
    \item \textbf{关键任务}:文献收集与综述、研究方法确定、研究计划制定。
\end{itemize}

\subsection{数据收集阶段}

\begin{itemize}
    \item \textbf{目标}:收集全面且准确的数据和案例,为后续分析提供基础。
    \item \textbf{关键任务}:数据收集、案例整理、初步数据处理。
\end{itemize}

\subsection{数据分析阶段}

\begin{itemize}
    \item \textbf{目标}:通过定性和定量分析,揭示开放式创新策略的实施效果。
    \item \textbf{关键任务}:数据分析、结果总结、图表制作。
\end{itemize}

\subsection{撰写报告阶段}

\begin{itemize}
    \item \textbf{目标}:完成研究报告的撰写,并进行初步审阅和修改。
    \item \textbf{关键任务}:报告撰写、内部审阅、修改完善。
\end{itemize}

\subsection{评审与修改阶段}

\begin{itemize}
    \item \textbf{目标}:根据反馈意见,进一步完善研究报告,准备答辩材料。
    \item \textbf{关键任务}:修改报告、准备答辩、最终定稿。
\end{itemize}

\section{研究风险及应对措施}

在研究过程中,可能面临一些风险和挑战,如数据收集困难、分析方法不当等。为此,制定以下应对措施:

\begin{itemize}
    \item \textbf{数据收集困难}:多渠道收集数据,确保数据的全面性和准确性。必要时,联系相关领域专家或企业获取支持。
    \item \textbf{分析方法不当}:在数据分析过程中,定期与导师和专家讨论,确保分析方法的科学性和合理性。必要时,调整分析方法。
    \item \textbf{时间管理问题}:制定详细的时间计划,严格按照计划执行。定期检查进度,确保研究按时完成。
\end{itemize}

通过详细的研究计划和有效的应对措施,确保本研究的顺利进行和高质量完成。


% 其他部分
\backmatter

% 参考文献
\bibliography{ref/refs}  % 参考文献使用 BibTeX 编译
% \printbibliography       % 参考文献使用 BibLaTeX 编译

% 附录
% 本科生需要将附录放到声明之后,个人简历之前
% \appendix
% % !TeX root = ../thuthesis-example.tex

\begin{survey}
\label{cha:survey}

\title{Title of the Survey}
\maketitle


\tableofcontents


本科生的外文资料调研阅读报告。


\section{Figures and Tables}

\subsection{Figures}

An example figure in appendix (Figure~\ref{fig:appendix-survey-figure}).

\begin{figure}
  \centering
  \includegraphics[width=0.6\linewidth]{example-image-a.pdf}
  \caption{Example figure in appendix}
  \label{fig:appendix-survey-figure}
\end{figure}


\subsection{Tables}

An example table in appendix (Table~\ref{tab:appendix-survey-table}).

\begin{table}
  \centering
  \caption{Example table in appendix}
  \begin{tabular}{ll}
    \toprule
    File name       & Description                                         \\
    \midrule
    thuthesis.dtx   & The source file including documentaion and comments \\
    thuthesis.cls   & The template file                                   \\
    thuthesis-*.bst & BibTeX styles                                       \\
    thuthesis-*.bbx & BibLaTeX styles for bibliographies                  \\
    thuthesis-*.cbx & BibLaTeX styles for citations                       \\
    \bottomrule
  \end{tabular}
  \label{tab:appendix-survey-table}
\end{table}


\section{Equations}

An example equation in appendix (Equation~\eqref{eq:appendix-survey-equation}).
\begin{equation}
  \frac{1}{2 \uppi \symup{i}} \int_\gamma f = \sum_{k=1}^m n(\gamma; a_k) \mathscr{R}(f; a_k)
  \label{eq:appendix-survey-equation}
\end{equation}


\section{Citations}

Example\cite{dupont1974bone} citations\cite{merkt1995rotational} in appendix
\cite{dupont1974bone,merkt1995rotational}.


% 默认使用正文的参考文献样式;
% 如果使用 BibTeX,可以切换为其他兼容 natbib 的 BibTeX 样式。
\bibliographystyle{unsrtnat}
% \bibliographystyle{IEEEtranN}

% 默认使用正文的参考文献 .bib 数据库;
% 如果使用 BibTeX,可以改为指定数据库,如 \bibliography{ref/refs}。
\printbibliography

\end{survey}
       % 本科生:外文资料的调研阅读报告
% % !TeX root = ../thuthesis-example.tex

\begin{translation}
\label{cha:translation}

\title{书面翻译题目}
\maketitle

\tableofcontents


本科生的外文资料书面翻译。


\section{图表示例}

\subsection{图}

附录中的图片示例(图~\ref{fig:appendix-translation-figure})。

\begin{figure}
  \centering
  \includegraphics[width=0.6\linewidth]{example-image-a.pdf}
  \caption{附录中的图片示例}
  \label{fig:appendix-translation-figure}
\end{figure}


\subsection{表格}

附录中的表格示例(表~\ref{tab:appendix-translation-table})。

\begin{table}
  \centering
  \caption{附录中的表格示例}
  \begin{tabular}{ll}
    \toprule
    文件名          & 描述                         \\
    \midrule
    thuthesis.dtx   & 模板的源文件,包括文档和注释 \\
    thuthesis.cls   & 模板文件                     \\
    thuthesis-*.bst & BibTeX 参考文献表样式文件    \\
    thuthesis-*.bbx & BibLaTeX 参考文献表样式文件  \\
    thuthesis-*.cbx & BibLaTeX 引用样式文件        \\
    \bottomrule
  \end{tabular}
  \label{tab:appendix-translation-table}
\end{table}


\section{数学公式}

附录中的数学公式示例(公式\eqref{eq:appendix-translation-equation})。
\begin{equation}
  \frac{1}{2 \uppi \symup{i}} \int_\gamma f = \sum_{k=1}^m n(\gamma; a_k) \mathscr{R}(f; a_k)
  \label{eq:appendix-translation-equation}
\end{equation}


\section{文献引用}

附录\cite{dupont1974bone}中的参考文献引用\cite{merkt1995rotational}示例
\cite{dupont1974bone,merkt1995rotational}。


\appendix

\section{附录}

附录的内容。


% 书面翻译的参考文献
% 默认使用正文的参考文献样式;
% 如果使用 BibTeX,可以切换为其他兼容 natbib 的 BibTeX 样式。
\bibliographystyle{unsrtnat}
% \bibliographystyle{IEEEtranN}

% 默认使用正文的参考文献 .bib 数据库;
% 如果使用 BibTeX,可以改为指定数据库,如 \bibliography{ref/refs}。
\printbibliography

% 书面翻译对应的原文索引
\begin{translation-index}
  \nocite{mellinger1996laser}
  \nocite{bixon1996dynamics}
  \nocite{carlson1981two}
  \bibliographystyle{unsrtnat}
  \printbibliography
\end{translation-index}

\end{translation}
  % 本科生:外文资料的书面翻译
% % !TeX root = ../thuthesis-example.tex

\chapter{补充内容}

附录是与论文内容密切相关、但编入正文又影响整篇论文编排的条理和逻辑性的资料,例如某些重要的数据表格、计算程序、统计表等,是论文主体的补充内容,可根据需要设置。

附录中的图、表、数学表达式、参考文献等另行编序号,与正文分开,一律用阿拉伯数字编码,
但在数码前冠以附录的序号,例如“图~\ref{fig:appendix-figure}”,
“表~\ref{tab:appendix-table}”,“式\eqref{eq:appendix-equation}”等。


\section{插图}

% 附录中的插图示例(图~\ref{fig:appendix-figure})。

\begin{figure}
  \centering
  \includegraphics[width=0.6\linewidth]{example-image-a.pdf}
  \caption{附录中的图片示例}
  \label{fig:appendix-figure}
\end{figure}


\section{表格}

% 附录中的表格示例(表~\ref{tab:appendix-table})。

\begin{table}
  \centering
  \caption{附录中的表格示例}
  \begin{tabular}{ll}
    \toprule
    文件名          & 描述                         \\
    \midrule
    thuthesis.dtx   & 模板的源文件,包括文档和注释 \\
    thuthesis.cls   & 模板文件                     \\
    thuthesis-*.bst & BibTeX 参考文献表样式文件    \\
    thuthesis-*.bbx & BibLaTeX 参考文献表样式文件  \\
    thuthesis-*.cbx & BibLaTeX 引用样式文件        \\
    \bottomrule
  \end{tabular}
  \label{tab:appendix-table}
\end{table}


\section{数学表达式}

% 附录中的数学表达式示例(式\eqref{eq:appendix-equation})。
\begin{equation}
  \frac{1}{2 \uppi \symup{i}} \int_\gamma f = \sum_{k=1}^m n(\gamma; a_k) \mathscr{R}(f; a_k)
  \label{eq:appendix-equation}
\end{equation}


\section{参考文献}

附录中的参考文献示例(\cite{carlson1981two} 和 \cite{carlson1981two,taylor1983scanning,taylor1981study})。

\printbibliography


% 致谢
% \input{data/acknowledgements}

% 声明
% 本科生开题报告不需要
% \statement
% 将签字扫描后的声明文件 scan-statement.pdf 替换原始页面
% \statement[file=scan-statement.pdf]
% 本科生编译生成的声明页默认不加页脚,插入扫描版时再补上;
% 研究生编译生成时有页眉页脚,插入扫描版时不再重复。
% 也可以手动控制是否加页眉页脚
% \statement[page-style=empty]
% \statement[file=scan-statement.pdf, page-style=plain]

% 个人简历、在学期间完成的相关学术成果
% 本科生可以附个人简历,也可以不附个人简历
% % !TeX root = ../thuthesis-example.tex

\begin{resume}

  \section*{个人简历}

  197× 年 ×× 月 ×× 日出生于四川××县。

  1992 年 9 月考入××大学化学系××化学专业,1996 年 7 月本科毕业并获得理学学士学位。

  1996 年 9 月免试进入清华大学化学系攻读××化学博士至今。


  \section*{在学期间完成的相关学术成果}

  \subsection{学术论文}

  \begin{achievements}
    \item Yang Y, Ren T L, Zhang L T, et al. Miniature microphone with silicon-based ferroelectric thin films[J]. Integrated Ferroelectrics, 2003, 52:229-235.
    \item 杨轶, 张宁欣, 任天令, 等. 硅基铁电微声学器件中薄膜残余应力的研究[J]. 中国机械工程, 2005, 16(14):1289-1291.
    \item 杨轶, 张宁欣, 任天令, 等. 集成铁电器件中的关键工艺研究[J]. 仪器仪表学报, 2003, 24(S4):192-193.
    \item Yang Y, Ren T L, Zhu Y P, et al. PMUTs for handwriting recognition. In press[J]. (已被Integrated Ferroelectrics录用)
  \end{achievements}


  \subsection{专利}

  \begin{achievements}
    \item 任天令, 杨轶, 朱一平, 等. 硅基铁电微声学传感器畴极化区域控制和电极连接的方法: 中国, CN1602118A[P]. 2005-03-30.
    \item Ren T L, Yang Y, Zhu Y P, et al. Piezoelectric micro acoustic sensor based on ferroelectric materials: USA, No.11/215, 102[P]. (美国发明专利申请号.)
  \end{achievements}

  \subsection*{3  奖项}

  \begin{achievements}
    \item 任天令, 杨轶, 朱一平, 等. 硅基铁电微声学传感器畴极化区域控制和电极连接的方法: 中国, CN1602118A[P]. 2005-03-30.
    \item Ren T L, Yang Y, Zhu Y P, et al. Piezoelectric micro acoustic sensor based on ferroelectric materials: USA, No.11/215, 102[P]. (美国发明专利申请号.)
  \end{achievements}

\end{resume}


% 指导教师/指导小组评语
% 本科生不需要
% % !TeX root = ../thuthesis-example.tex

% \begin{comments}
\begin{comments}[name = {基础导师学术评语}]
% \begin{comments}[name = {Comments from Thesis Supervisor}]
% \begin{comments}[name = {Comments from Thesis Supervision Committee}]

  论文提出了……

\end{comments}


\begin{comments}[name = {临床导师学术评语}]

  论文提出了……

\end{comments}


% 答辩委员会决议书
% 本科生不需要
% \input{data/resolution}

% 本科生的综合论文训练记录表(扫描版)
% \record{file=scan-record.pdf}

\end{document}
